\documentclass[../main.tex]{subfiles}

\begin{document}

\section{Simulation overview}

Let $N_i$ denote the $M$ nearest neighbors of $i$. The following equations fully describe the
simulation:
\begin{multicols}{2}

\begin{equation}
    h_a = \frac{1}{\sigma} \max_{b \in N_i} ||\bm{r_a} - \bm{r_b}||
\end{equation}

\begin{equation}
    h_{ab} = \min(h_a, h_b)
\end{equation}

\begin{equation}
    W_{ab} = \frac{1}{h_{ab}^3 \, \pi^{\frac{3}{2}}} e^{-(\frac{||\bm{r_a} - \bm{r_b}||}{h_{ab}})^2}
\end{equation}

\begin{equation}
    \nabla_a W_{ab} = -2 \frac{\bm{r_a} - \bm{r_b}}{h_{ab}^2} W_{ab}
\end{equation}

\begin{equation}
    \rho_a = \max \left( \mathtt{\epsilon_{f64}}, \sum_{b \in N_a} m \, W_{ab} \right)
\end{equation}

\begin{equation}
    P_a = \frac{2 \, E_a \, \rho_a}{3 \, m}
\end{equation}

\begin{equation}
    \bm{g_a} = \sum_{b \in \{1 \dots n\} \setminus a} G m \frac{\bm{r_b} - \bm{r_a}}{||\bm{r_b} - \bm{r_a}||^3}
\end{equation}

\begin{equation}
    \frac{d\bm{v_a}}{dt}
    = \bm{g} - \sum_{b \in N_a} m \, (\frac{P_a}{\rho_a^2} + \frac{P_b}{\rho_b^2}) \, \nabla_a W_{ab}
\end{equation}

\begin{equation}
    \frac{dE_a}{dt}
    = \frac{m}{2} \sum_{b \in N_a} m \, (\frac{P_a}{\rho_a^2} + \frac{P_b}{\rho_b^2}) (\bm{v_a} - \bm{v_b})
    \cdot \nabla_a W_{ab}
\end{equation}

\begin{equation}
    \frac{d\bm{r_a}}{dt}
    = \bm{v_a} + \epsilon \sum_{b \in N_a} m \, \frac{2 (\bm{v_b} - \bm{v_a})}{\rho_a + \rho_b} \, W_{ab}
\end{equation}

\end{multicols}

Where $\mathtt{\epsilon_{f64}} = 2^{-1022}$ is used as shorthand for making $\frac{0}{0} = 0$ in
several interpolants. $G = 6.674 \cdot 10^{-11}$ is the gravitational constant with SI units
$\SI{}{\meter\cubed\per\kilo\gram\per\second\squared}$.

The dispersion factor $\epsilon$, the spatially constant particle mass $m$, the smoothing length
cutoff $\sigma$, the initial conditions $\bm{r_a}$, $\bm{v_a}$ and $E_a$, as well as the number of
particles $n$, the number of neighbors $M$ and the time step $\Delta t$ are all configured on a per simulation basis in the
program.

The numerical integration uses the following equations:
\begin{equation}
  \bm{r_a} := \bm{r_a}
    + \frac{d\bm{r_a}}{dt} \Delta t
    + \frac{d\bm{v_a}}{dt} \frac{\Delta t^2}{2}
\end{equation}
\begin{equation}
    \bm{v_a} := \bm{v_a} + \frac{d\bm{v_a}}{dt} \Delta t
\end{equation}
\begin{equation}
    E_a := E_a + \frac{dE_a}{dt} \Delta t
\end{equation}


The code is written in Rust and is available at
\href{https://github.com/lokegustafsson/binary-accretion}{github.com/lokegustafsson/binary-accretion}.

\subsection{Initial condition generation}
The particles all have to be given a position, a velocity and a thermal energy. The $n$ particles
are placed randomly within a sphere of a given radius, with one of the following density profiles:

\begin{itemize}
    \item Uniform, i.e. $\rho(r) \propto 1$
    \item Inverse linear, i.e. $\rho(r) \propto \frac{1}{r}$
    \item Inverse square, i.e. $\rho(r) \propto \frac{1}{r^2}$
\end{itemize}

The particles are given the required velocity for rigid-body rotation with a set rotational
period. The particles are all initially given the same temperature, which since they all have the
same mass and molar mass gives them the same initial thermal energy.

\section{Data collected}
In addition to visually displaying the particles, the simulation computes and outputs the following:
\begin{enumerate}
    \item Simulation iterations (ticks) per second
    \item Net movement, i.e. total momentum per unit mass
    \item Total gravitational potential energy
    \item Total kinetic energy
    \item Total thermal energy
    \item Arithmetic mean of particle temperatures
    \item Arithmetic mean of particle pressures
\end{enumerate}

\section{Specific simulations run}

\subsection{Reference simulation}
Having a reference simulation allows us to perform $n+1$ rather than $2n$ simulations to make $n$
binary comparisons between simulation parameters. The following standard parameters were chosen to
roughly match the properties of a protostellar nebulae:
\\

\begin{tabular}{ll}

\multicolumn{2}{l}{Computational} \\
    \qquad Particle count:                     & $2 \, 000$ \\
    \qquad $\Delta t$:                         & $500$ years \\
\multicolumn{2}{l}{General initial conditions} \\
    \qquad Initial radius:                     & $10 \, 000$ AU (about $0.15$ light years)\\
    \qquad Initial rotational period:          & $10$ million years \\
    \qquad Initial density curve:              & Uniform \\
\multicolumn{2}{l}{Gravity} \\
    \qquad Total mass:                         & $1$ solar mass \\
\multicolumn{2}{l}{SPH} \\
    \qquad Initial temperature:                & $\SI{5}{\kelvin}$ \\
    \qquad Molar mass:                         & $\SI{2.016}{\gram\per\mole}$ (hydrogen gas) \\
    \qquad Neighbor count $|N_i|$:             & $30$ \\
    \qquad Smoothing distance cutoff $\sigma$:        & $2.0$ \\
    \qquad Velocity averaging $\epsilon$:      & $100\%$ \\

\end{tabular}

\subsection{Very high initial temperature}

\begin{tabular}{ll}
\multicolumn{2}{l}{Modifications} \\
\qquad Initial temperature:                    & $\SI{1000}{\kelvin}$ \\
\end{tabular}
\\

\subsection{Rotating}

\begin{tabular}{ll}
\multicolumn{2}{l}{Modifications} \\
    \qquad Initial rotational period:          & $500 \, 000$ years \\
\end{tabular}
\\

\subsection{Rotating, with a dense core}

\begin{tabular}{ll}
\multicolumn{2}{l}{Modifications} \\
    \qquad Initial rotational period:          & $500 \, 000$ years \\
    \qquad Initial density curve:              & Inverse quadratic \\
\end{tabular}
\\


\end{document}
