\documentclass{report}

\usepackage{amsmath}
\usepackage{amssymb}
\usepackage{bm}
\usepackage{graphicx}
\usepackage{titlesec}
\usepackage{subfiles}
\usepackage{csquotes}
\usepackage{setspace}

\newcommand{\dwrt}{\,\mathrm{d}}

\usepackage[hidelinks]{hyperref}
\hypersetup{linktoc=all, }

\usepackage[style=verbose-trad2]{biblatex}
\bibliography{references.bib}

% Title page
\makeatletter
\newcommand{\subtitle}[1]{\def\@subtitle{#1}}
\newcommand{\supervisor}[1]{\def\@supervisor{#1}}
\makeatother
\title{Simulating gas with self-gravitational interaction using smoothed-particle hydrodynamics}
\subtitle{}
\author{Loke Gustafsson}
\supervisor{Mats Gustafsson}
\date{\today}

% Chapter formatting
\titleformat{\chapter}
    {\Large\bfseries} % format
    {}                % label
    {0pt}             % sep
    {\huge}           % before-code

% Document
\begin{document}
\begin{spacing}{1.5}

\subfile{titlepage.tex}

\chapter*{Abstract}
    \subfile{abstract.tex}

\tableofcontents

\chapter{Introduction}
    \subfile{introduction.tex}

\chapter{Theory}
    \section{Adiabatic, inviscid and compressible flow in an ideal gas}
    \subfile{theory/fluid_dynamics.tex}

    \section{Gravity}
    \subfile{theory/gravity.tex}

    \section{Smoothed-particle hydrodynamics}
    \subfile{theory/smoothed_particle_hydrodynamics.tex}

    \section{Computing particle neighborhoods}
    \subfile{theory/neighborhoods.tex}

\chapter{Method}
    \subfile{method.tex}

\chapter{Result}
    \subfile{result.tex}

\chapter{Discussion}
    \subfile{discussion.tex}

\printbibliography[heading=bibintoc,title={References}]

\end{spacing}
\end{document}
