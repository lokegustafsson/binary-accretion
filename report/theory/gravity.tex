\documentclass[../main.tex]{subfiles}

\begin{document}

Gravity can be described using Newton's law of gravity. In vector form, with a finite number of
sources:

\begin{equation}
    \bm{g} = \sum_b \frac{G m_b (\bm{r_b} - \bm{r})}{||\bm{r_b} - \bm{r}||^3}
\end{equation}

Since the work done by the gravitational field is path-independent we can define a gravitional
potential field:

\begin{equation}
    \bm{g} = - \nabla \phi
\end{equation}

As gravity is an attractive force increasing with smaller distances it makes
sense to define the potential energy as the work done by bringing the particles from infinity to
their current positions.

For a single pair of particles this is readily computed:

\begin{equation}
    \int_{\infty}^{||\bm{r_a} - \bm{r_b}||} \frac{G m_a m_b}{r^2} dr = \frac{-G m_a m_b}{||\bm{r_a} -
    \bm{r_b}||}
\end{equation}

As the force is just a sum between all pairs of particles, a system's total potential energy is the sum of the
potential energy between all pairs of particles, divided by two to avoid double counting.

\begin{equation}
    \frac{1}{2} \sum_{\substack{a, b \\ a \ne b}} \frac{-G m_a m_b}{|\bm{r_a} - \bm{r_b}|}
\end{equation}

Newtonian gravity deviates from general relativity at high velocities and high gravitational
potentials. More precisely, the deviation is small when the dimensionless quantities $(\frac{v}{c})^2$
and $\frac{\phi}{c^2}$ are both much less than one. \autocite{wikipedia_newtons_law}

These sums can be computed by summing over all the particles. Unlike the interpolants described in
the next section, the effect of far-away particles is not negligible, making it more difficult to
compute efficiently. There are many subquadratic methods for the $N$-body problem, all generally
based on grouping clusters of distant particles into single sources. In particular, the fast
multipole method (FMM) computes the forces on all particles in $O(N \log (1 / \epsilon) )$, where
$\epsilon$ is the tolerated relative error. \autocite{shortcourse_fmm} In this work, however, a
quadratic direct sum is used due to time constraints.

\end{document}
