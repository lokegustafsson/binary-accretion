\documentclass[../main.tex]{subfiles}

\begin{document}

The physical interaction within a gas is governed by the conservation of mass, momentum and energy.
These laws can, however, due to a gas's continous nature not simply be expressed as sums over a
finite collection of physical objects. Instead, we use fields. The aforementioned conservations laws
are adapted to continous fluids in the Euler equations: \autocite{wikipedia_euler_equations}

\begin{equation}
    \frac{D\rho}{Dt}
    = - \rho \nabla \cdot \mathbf{u}
\end{equation}

\begin{equation}
    \frac{D\mathbf{u}}{Dt}
    = - \frac{\nabla P}{\rho} + \mathbf{g}
\end{equation}

\begin{equation}
    \frac{Du}{Dt}
    = - \frac{P}{\rho} \nabla \cdot \mathbf{u}
\end{equation}

Where $\rho$, $P$, $u$ are density, pressure and specific internal energy respectively,
$\mathbf{u}$ is the velocity flow field, $\mathbf{g}$ denotes body accelerations such as gravity
and magnetism and $\nabla$ is the del operator.

$\frac{Dy}{Dt}$ is the material derivative, capturing both changes within matter and changes due to its
movement: $\frac{Dy}{Dt} = \frac{\partial y}{\partial t} + \mathbf{u} \cdot \nabla y$. In a
Lagrangian simulation (such as in smoothed-particle hydrodynamics) this is already accounted for
through the movement of simulation particles.

The relation between the pressure, density, and specific internal energy can easily be expressed for
an ideal gas:

\begin{equation}
    \frac{2E}{3} = nRT = PV \implies P = \frac{2}{3} u \rho
\end{equation}

\end{document}
