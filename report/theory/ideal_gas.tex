\documentclass[../main.tex]{subfiles}

\begin{document}

Move all fluid-dynamics material from SPH into here. Universal gas law but also euler's
equations in fluid dynamics.

This work uses the Euler equation for momentum balance of a compressible inviscid fluid:

\begin{equation}
    \frac{d\bm{v_a}}{dt}
    = - \frac{1}{\rho_a} (\nabla P)_a + \bm{g}_a
\end{equation}

A mass balance equation is not necessary as the density is interpolated from the particle masses in
each iteration.

The only external force considered is gravity, as magnetism, radiative transfer and special
relativity are all ignored in this work: (The gravitational field is the negative gradient of the
gravitational potential field)

\begin{equation}
    \bm{g}_a
    = -\nabla_a\int\frac{-G\rho(\bm{r'})}{|\bm{r_a} - \bm{r'}|} d\bm{r'}
    = \int\frac{G\rho(\bm{r'})(\bm{r_a}-\bm{r'})}{|\bm{r_a}-\bm{r'}|^3} d\bm{r'}
\end{equation}

This integral can be computed by summing over all the particles. Note that this is not an
interpolation integral, and computing it naively for all the particles would take quadratic time
regardless of the kernel function. For this reason, we use a more sophisticated tree structure to
compute the gravitational acceleration.

Calculating the gradient of the pressure requires us to compute the pressure at the particles using
the ideal gas law.

\begin{equation}
    \frac{2E}{3} = nRT = PV = \frac{Pm}{\rho}
\end{equation}

Giving us the pressure at a particle

\begin{equation}
    P_a = \frac{2 E_a}{3 m_a} \rho_a = \tau_a \rho_a
\end{equation}

The pressure factor $\tau_a = \frac{2E_a}{3m_a}$ is constant in this paper's
isothermal simulation.
\end{document}
