% !TEX root = ../main.tex
\documentclass[../main.tex]{subfiles}

\begin{document}
\subsection{Introduction}
Smoothed-particle hydrodynamics (SPH) is a method in computational fluid dynamics used to numerically solve a system of differential equations. Insert history.

SPH is a particle-based, Lagrangian method. This confers several advantages over mesh-based, Eulerian methods in astrophysical problems such as those studied in this paper:
\begin{enumerate}
    \item The lack of boundary conditions.
    \item Large regions of low density
    \item Adaptive resolutions
\end{enumerate}

\subsection{Fundamentals}
For a given vector or scalar field $A$ we define the integral interpolant $A_I \approx A$ as

\begin{equation}
    A_I(\bm{r}) = \int A(\bm{r}) W(\bm{r} - \bm{r'}, h) d\bm{r'}
\end{equation}

where $\bm{r}$ and $\bm{r'}$ are position vectors with units $m$ and $h$ is a scalar ``smoothing length'' with units $m$. The (scalar) kernel function $W(\bm{r}, h)$ has units $m^{-3}$ and satisfies

\begin{equation}
    \int W(\bm{r}, h) d\bm{r} = 1
\end{equation}

and

\begin{equation}
    \lim_{h\to0} W(\bm{r}, h) d\bm{r} = \delta(x) = \begin{cases}
        +\infty & \text{for } x = 0 \\
        0 & \text{otherwise} \end{cases}
\end{equation}

The field $A_I$ is a smoothed version of $A$, with a granularity dependent on $h$. The core idea in SPH is to approximate the integral interpolant with a finite sum over the particles, and so we define the summation interpolant $A_S \approx A_I$ as

\begin{equation}
    A_S(\bm{r}) = \sum_{b} V_b A(b) W(\bm{r} - \bm{r_b}, h)
\end{equation}

where the sum is over all the SPH particles. The particles each have a position $\bm{r}$, velocity $\bm{v}$, mass $m$ and thermal energy $u$. All other fields have to be computed and interpolated from these, the most fundamental of which is the density.

The density field is particularly important due to the particles having known masses, and the summation interpolant using the volumes of the particles. Writing the volume $V_b$ as $\dfrac{m_b}{\rho(\bm{r_b})}$ allows us to talk about the density rather than the volume:

\begin{equation} \label{eq:2.5}
    A_S(\bm{r}) = \sum_{b} \frac{m_b}{\rho(\bm{r_b})} A(\bm{r_b}) W(\bm{r} - \bm{r_b}, h)
\end{equation}

additionally giving us the density interpolant for free

\begin{equation}
    \rho_S(\bm{r}) = \sum_{b} \frac{m_b}{\rho(\bm{r_b})} \rho(\bm{r_b}) W(\bm{r} - \bm{r_b}, h) = \sum_{b} m_b W(\bm{r} - \bm{r_b}, h)
\end{equation}

At this point we drop the $I$ and $S$ subscripts for the interpolants since this would be unnecessarily verbose and intent will be clear from context. This allows us to shorten $A_S(\bm{r_b})$ to $A_b$. We will also write $W(\bm{r_a} - \bm{r_b}, h)$ as $W_{ab}$ and let $\nabla_a$ denote the gradient with respect to the space coordinates of particle $a$.

\subsection{Operators}
In order to solve the differential equations we need a way to compute gradients, divergence and curls within the SPH framework. More specifically, we need a way to compute $\nabla A$, $\nabla\cdot\bm{F}$ and $\nabla\times\bm{F}$, where $A$ and $\bm{F}$ are arbitrary scalar and vector fields respectively.

Using (\ref{eq:2.5}) we can write $(\nabla A)_a$ as

\begin{equation}
    (\nabla A)_a = (\nabla_a A)_a =
    \sum_b \frac{m_b}{\rho_b} A_b \nabla_a W_{ab} =
    \sum_b \frac{m_b A_b \nabla_a W_{ab}}{\sum_c m_c W_{bc}}
\end{equation}

giving us the gradient of any field for which we can compute $A_b$.

However, it turns out \autocite{monaghan1992} that this nested sum decreases precision. Instead, the usual approach is to calulate the interpolated $(\nabla(\rho A))_a$ and then retrieve $A$ through $\rho\nabla A = \nabla(\rho A) - \nabla\rho A$

\end{document}