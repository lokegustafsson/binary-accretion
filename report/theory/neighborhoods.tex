\documentclass[../main.tex]{subfiles}

\begin{document}

The SPH interpolants are of the form
\[ F_a = \sum_b f(a, b) W_{ab} \qquad\textnormal{and}\qquad G_a = \sum_b g(a, b) \nabla_a W_{ab} \]
Computing these sums exactly with the gaussian kernel takes quadratic time, which severely limits
the number of particles that can be practically simulated. Luckily, the influence of particles
decays exponentially with distance, allowing us to neglect particles $b$ for which
$||\bm{r_a} - \bm{r_b}|| > \sigma \,  h_{ab}$ for some constant $\sigma$. The set of particles $b$
for which $||\bm{r_a} - \bm{r_b}|| \le f(a, b)$ cannot be computed efficiently for most $f$.

Summing over $b$ where $||\bm{r_a} - \bm{r_b}|| \le \sigma' h_a$ is much easier algorithmically as
it allows methods from computational geometry to be applied. It does however not uphold symmetry
properties --- momentum in particular is not conserved. To fix this we need to choose a $h_{ab}$
such that $\sigma h_{ab} \le \sigma' h_a$, which in turn requires
$h_{ab} = \frac{\sigma'}{\sigma} \min(h_a, h_b)$. $\sigma$ is already encapsulated in the choice
of $h_a$ and can be assumed to equal $\sigma'$. Additionally, to conserve the symmetric properties
exactly it is not enough for the influence to decay exponentially. Rather, we need $W(r, h) = 0$
for $r \ge \sigma'h$. This modification is easily made to the kernel function while influencing the
simulation negligibly. Finally, this can lead to $F_a$ and $G_a$ being zero instead of small but
non-zero. This causes problems when dividing by density, as $\frac{0}{0}$ is indeterminate. We can
fix this by setting $\rho_a := max(\rho_a, \epsilon)$ where $\epsilon$ is the smallest positive
number expressible by the floating-point numbers used.

The set of relevant neighbors ($\left\{ b \mid ||\bm{r_a} - \bm{r_b}|| \le \sigma' h_a \right\}$)
needs to be computed and should have a similar size for all the particles in order to minimize the
error given the computational power. We solve this by computing all-$m$-nearest-neighbors for a
constant $m$ and then choosing $h_a = \frac{\max_b ||\bm{r_a} - \bm{r_b}||}{\sigma'}$.
\\

\textsc{All-$m$-nearest-neighbors}
\\
Given $n$ points $\bm{r}_1, \bm{r}_2 \dotsc \bm{r}_n$ and an integer $m$, find subsets
$T_1, T_2 \dotsc T_n \subset \{ 1 \dots n \}$ such that $i \notin T_i$ and
$\displaystyle \max_{j \in T_i} ||\bm{r}_i - \bm{r}_j||$ is minimized.
\\

For euclidian distance in $\mathbb{R}^3$ this can be solved in $O(m n \log n)$ worst case using the
algorithm described by \textcite{neighbors}. It is also possible that an adaptation of one of the
all-nearest-neighbors algorithms by \textcite{random_neighbors} might prove faster in practice on
non-adversarial inputs.

This paper uses the naive $O(n^2 \log m)$ algorithm: Looping through all other points for every
point while keeping the best candidates in a priority queue. This was found to be faster than
adapting existing implementations for the $m$-nearest-neighbors problem on
\href{https://crates.io}{\texttt{crates.io}}.

\end{document}
