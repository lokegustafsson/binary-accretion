\documentclass[../main.tex]{subfiles}

\begin{document}

The results show that:
\begin{enumerate}
    \item Momentum is conserved, up to errors induced by limited floating-point precision.
    \item The sum of gravitational potential, kinetic and thermal energy is not a conserved
        quantity.
    \item The average temperature strongly correlates with the average pressure
    \item The reference parameters lead to pulsation in the gas. The pulsation period is roughly
        constant over the three expansions and contractions observed in the simulation. The mean
        temperature in the valleys is roughly constant: $4.48$, $4.62$ and $4.67$, while the peak
        temperature decreases sharply from $7.57$ to $6.56$ to $5.80$ in the three contractions. No
        dense core develops.
    \item A high initial temperature leads to massive expansion. Roughly $30\%$ of the thermal
        energy is converted to kinetic energy in the first $50 \, 000$ years, and the overwhelming
        majority of the total acceleration during the simulation occurs in that period.
        Additionally, visual observation indicates that the initially internal particles during that
        period are mostly accelerated inwards. This raises questions about how their velocity is
        redirected.
    \item The collapse of rotating particles into a flat disk is analogous to the first collapse of
        the reference simulation. The subsequent expansion then occurs in all directions, but since
        the rotation has given a head start to expansion in the rotatonal plane the particles expand
        in a flattened shape.
    \item Rotational with a dense core is similar to rotational without a dense core. The main
        difference is that the core can maintain its mass despite the rotation. That the core does
        not rotate much faster than the initial rotational implies that angular momentum has not
        been transfered from the initially outer layers. This likely means that very few particles
        initially in the outer layers end up in the stable core at the end of the simulation.
\end{enumerate}

That momentum is conserved and that temperature correlates with pressure implies that the simulation
roughly corresponds to the behavior of a real gas. It is more problematic that no dense core
develops in the reference simulation or the rotating simulation of uniform density, and that
particles are not accreted to the core in the rotating simulation with a dense core. A mechanism
driving this is clearly missing from this papers computational model. It is reasonable to assume
that the main missing mechanism is viscosity; the lack of accretion above all else suggests a lack
of dissipation of kinetic energy. Implementing viscosity would therefore be the most important step
towards improving the simulation.

Additionally, the particle nature of SPH makes the final simulation more particle-like than a real
gas. This can for example be seen through the fast ejection of single particles involved in
gravitational close encounters at the beginning of the simulations. It is however likely that this
affects the vast majority of the particles only negligibly.

Finally, the choice of $|N_i| = 30$ was made arbitrarily for computational reasons - yet it
determines the resolution and distance at which the particles interact with each other as a gas.
Further investigating this parameter and through it the smoothing length is crucial for
understanding smoothed-particle hydrodynamics.

\end{document}
